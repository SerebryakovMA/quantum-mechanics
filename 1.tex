
\newpage
\chapter{Модель атома Бора}
\par    В 1904 году Д.Д. Томпсоном была представлена первая модель атома. Открытию предшествовало обнаружение электронов, а после формулировки гипотезы было обнаружено атомное ядро. Томпсон говорил, что атом является равномерно распределенным по всему объему зарядом со знаком +. Положительно заряженное «облако» содержит внутри небольшие электроны с отрицательным зарядом, расположение которых определено случайно.  Общий заряд атома нейтрален, что обусловлено равенством по модулю суммарного заряда электронов и заряда «облака». 
\par Модель Томпсона - объяснение излучения атомов. Однако формулы определенных химических элементов описали их спектры, но противоречили рассматриваемой модели. Не получилось объяснить дискретный характер, которым обладают спектры атомов (!). Проблема также заключалась в описании устойчивости атома (!). Представленная модель не могла охарактеризовать рентгеновское и гамма-излучения, которые испускают атомы. Отсутствовали пояснения относительно определения размеров атома. Модель противоречила опытам, направленным на изучение того, как распределяется положительный заряд в атоме (!). 
\par В 1911 году была представлена более точная модель атома Резерфорда: положительный заряд расположен в малой области атома, а компенсирующие электроны окружают его. К такому утверждению ученый пришел в результате экспериментов по бомбардировке атомов. Не объясняет энергетическую устойчивость атома (!). Электрон вращается вокруг ядра, следовательно, имеет центростремительное ускорение, следовательно, излучает энергию, радиус вращения уменьшается и электрон должен упасть на ядро. А этого не происходит.
\par В 1913 году датский физик Н. Бор, проанализировав всю совокупность опытных фактов, пришел к выводу, что при описании поведения атомных систем следует отказаться от многих представлений классической физики. Он предположил, что сам факт существования атомов свидетельствует о том, что существуют минимальные "размеры", ближе которых к ядру электроны не могут подобраться. Бор вспоминал, что "будучи в Манчестере на стажировке у Резерфорда, он пришел к убеждению, что строение электронного роя в атоме управляется квантом действия – постоянной Планка". В статье "О строении атомов и молекул" он использовал гипотезу астрофизика Николсона, которая объясняла устойчивость "орбит" электронов кратностью их орбитального момента импульса величине постоянной Планка, уменьшенной в 2 пи. Это соответствовало требованию равенства длины орбиты целому числу длин волн де Бройля для электрона. 
\par Итак, энергия есть интеграл движения, а согласно планетарной теории существует еще один. Это угловой момент, гипотеза Николсона:
$$L_z=n \hbar$$
\par Пусть орбита круговая, тогда есть центростремительное ускорение, а уравнение движения выглядит следующим образом:
$$\frac{m \textit{v}^2}{r} = \frac{Z e^2}{r^2} $$
\par где $Z e$ - заряд ядра (само $Z$ - степень ионизации). Но по определению $L_z=m \textit{v} r=\frac{Z e^2}{\textit{v}}$, а полная энергия электрона \textit{E} равна сумме его кинетической и потенциальной энергий:
$$ E=\frac{m \textit{v}^2}{2}-\frac{Z e^2}{r} = \frac{Z e^2}{2r} - \frac{Z e^2}{r} = -\frac{Z e^2}{2r}=  -\frac{m \textit{v}^2}{2} = - \frac{m Z^2 e^4}{2L_z^2} $$
\par Т.к. $L_z$ дискретно, то есть на каждом шаге добавляется $\hbar$:  $ \Delta L_z = \hbar $ ,  если мы считаем эту добавку малой, то можно разложить энергию в Тейлора и получить:
$$\Delta E = \frac{m Z^2 e^4 \Delta L_z}{L_z^3} = \frac{\hbar m Z^2 e^4}{L_z^3}=\hbar \frac{\textit{v}}{r} $$
\par С одной стороны, $\Delta E = \hbar w$, где \textit{w} - частота излучения света порциями, с другой - в пределе малого $\hbar$ есть $\hbar \frac{\textit{v}}{r} = \hbar w$, где \textit{w} уже частота обращения электрона по орбите. Но эти частоты не могут быть равны, на самом деле, данным разложением мы пользоваться не имели права, т.к. изменение $ L_z $ может быть не мало, а малоблизкими изменениями энергии $n \shortrightarrow n+1$ ограничиваться нельзя, рассмотрим любые значения \textit{l} и \textit{n}, получив формулу Бальмера:
$$\hbar w_{l n}=\frac{m e^4}{2 \hbar^2} (\frac{1}{l^2}-\frac{1}{n^2}) $$
\par Константа для z=1 $R_и=\frac{m e^4}{2 \hbar^2}=13606 эВ$ называется ридберг. Полученная формула не имеет никакого отношения к частоте обращения, проблема была преодолена. Но данная теория не была последовательной. С одной стороны, она отвергает описание атома на основе классической физики, так как постулирует наличие стационарных состояний и правила квантования, непонятные с позиций механики и электродинамики. С другой стороны, для записи уравнения движения электрона по круговой орбите используются именно классические законы: второй закон Ньютона и закон Кулона. Несмотря на свои недостатки, теория Бора стала важнейшим этапом развития физики микромира. Спустя более чем десятилетие после создания первой квантово-механической модели атома водорода Бором была построена новая законченная и непротиворечивая квантово-механическая теория, в целом с успехом используемая до настоящего времени.