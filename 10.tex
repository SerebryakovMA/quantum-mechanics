\newpage
\chapter{Вычисление средних величин в квантовой механике}
\par Рассмотрим, как на языке $\hat{f}$-представления найти среднее. Пусть спектр дискретен: $\hat{f}\psi = \sum_{n} a_n f_n  \psi_n$, среднее можно найти по формуле $ \overline{f}=\sum_{n}f_n |a_n|^2$. Из условия нормировки ранее получали  $a_n = \sum_{m}a_m \int \psi_m \psi_n^* dq$, подставим это и получим интересную формулу:
$$ \hat{f} \psi = \int K(q{,}q\prime) \psi (q\prime) dq\prime \text{, где } K(q{,}q\prime)=\sum f_n \psi_n^* (q\prime) \psi_n (q)  $$
\par Отсюда следует, что всякий оператор представим в виде некого интеграла со своим ядром. Аналогично поступим с непрерывным спектром, пользуясь следствием нормировки (см билет 8): 
$\hat{f}\psi_f = f \cdot \psi_f $, $\psi (q)= \int a_f \psi_f df$, среднее можно найти как  $\overline{f} = \int \psi^* \hat{f} \psi dq$ //доделать ручками
