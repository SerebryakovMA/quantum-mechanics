\newpage
\chapter{Соотношение неопределенностей}
\par Неопределенность координаты и импульса есть среднеквадратичное отклонение: $\delta x = \sqrt{\overline{(x-\overline{x})^2}}$, $\delta p_x = \sqrt{\overline{(p_x-\overline{p_x})^2}}$.  $\overline{x}=\overline{p_x}=0$
$$\int^\infty_\infty |\alpha x \psi + \frac{d\psi}{dx}|^2 dx \geq 0 \text{, }\alpha \in \Re $$
\par $\int x^2|\psi|^2dx=(\delta x)^2$; $\int (x \frac{d\psi^*}{dx}\psi + x \psi^* \frac{d\psi}{dx})dx = \int x \frac{d|\psi|^2}{dx}dx= - \int |\psi|^2 dx = -1$
$$\int \frac{d\psi^*}{dx} \frac{d\psi}{dx}dx = - \int \psi^* \frac{d^2 \psi}{dx^2} dx = (\frac{\delta p_x}{\hbar})^2$$
\par Итого:
$$\alpha^2(\delta x)^2 - \alpha + (\frac{\delta p_x}{\hbar})^2 > 0$$
$$(\alpha \delta x -\frac{1}{2\delta x})^2 - \frac{1}{4(\delta x)^2} + (\frac{\delta p_x}{\hbar})^2 \geq 0$$
\par Т.к. вышеупомянутое должно быть верно при любых действительных значениях $\alpha$, то верно и для такого конкретного значения, что 1ая скобка занулится, а значит
$$(\frac{\delta p_x}{\hbar})^2 \geq \frac{1}{4(\delta x)^2} $$
$$ \delta p_x \cdot \delta x \geq \frac{\hbar}{2}$$
\par Последнее соотношение называют соотношением неопределенностей.
