\newpage
\chapter{Cвязь между стационарным и нестационарным уравнением Шрёдингера (нет в билетах этого года)}
\par Запишем УШ $i\hbar \frac{\partial \psi}{\partial t} = \hat{H}\psi$, где $\psi = \psi (q, t)$, т.е. эволюция системы отражается в эволюции $\psi$-функции. Средние значения вычисляются как
$$ \overline{f}=\int \psi^*(q, t)\underbrace{\hat{f_q}}_{\text{действует на q}}dq$$
\par Хотим перенести зависимость от времени от волновой функции на оператор, т.е. так называемое \textit{представление Шредингера}
$$  \overline{f}=\int \psi^*(q, 0)\hat{f_q}(t)\psi(q, 0)dq$$
\par Обачные классические уравнения $\dot{\vec{p}}=- \nabla U$ хотим преобразовать в $\hat{\dot{\vec{p}}} = - \nabla \hat{U} $  - \textit{Гейзенберговское представление}. Применение различных представлений - лишь вопрос удобства. Введем \textit{оператор эволюции} $\hat{S}=e^{-\frac{i}{\hbar} \hat{H}t}$ (в предположении, что оператор гамильтона от времени не зависит). Этот оператор действует на волновую функцию и переводит её в функцию без зависимости от времени $\hat{S}\psi_n(q, t)=e^{-\frac{i}{\hbar} E_n t} \psi_n (q)$ ($\psi_n$ - стационарные собственные функции $\hat{H}\psi_n = E_n \psi$ ). Переход между представлением Шредингера и Гейзенберга: $\psi(q, t)=\hat{S}\psi(q, 0) $.
\par \begin{theorem} Оператор эволюции унитарен

$$\hat{S}^+\hat{S}=1$$
\proof \par Получим временное уравнение для оператора эволюции. С этой целью рассмотрим УШ для стационарного состояния и учтем действие оператора эволюции на волновую функцию:
$$\hat{S}\psi_n(q, t)=e^{-\frac{i}{\hbar} E_n t} \psi_n (q)$$
$$\psi(q, t)=\hat{S}\psi(q, 0) $$
\par Получим
$$i\hbar \frac{\partial}{\partial t} \hat{S} \psi(q, 0)= \hat{H} \hat{S} \psi(q,0)$$
\par Поскольку это уравнение выполняется для любых $\psi(q, 0)$ (из гильбертова пространства), то $i\hbar \frac{\partial}{\partial t} \hat{S} = \hat{H} \hat{S}$. Воспользуемся временным уравнением операторов эволюции $\hat{S}$ и $\hat{S}^+$:
$$i\hbar \frac{\partial}{\partial t} \hat{S} = \hat{H} \hat{S} \text{, и } i\hbar\frac{\partial}{\partial t} \hat{S}^+ = - \hat{S}^+ \hat{H}$$
\par Рассмотрим производную по времени
$$i\hbar \frac{\partial}{\partial t}( \hat{S}^+ \hat{S})= i \hbar \frac{\partial \hat{S}^+}{\partial t} \hat{S} +  i \hbar \hat{S}^+\frac{\partial \hat{S}}{\partial t}= (-\hat{S}^+\hat{H}\hat{S}+\hat{S}^+\hat{H}+\hat{S})=0$$
\par В начальный момент времени оператор эволюции является единичным оператором и, следовательно, унитарным. Посколько производная по времени от произведения операторов $\hat{S}^+ \hat{S}$ равна нулю, то это произведение не зависит от времени и $\hat{S}^+ \hat{S}=\hat{I}$, т.е. $\hat{S}^+= \hat{S}^{-1}$ оператор эволюции унитарен.

\textbf{Теорема доказана.}
\end{theorem}
\par Итак, рассмотрим 
$$ \overline{f}(t) = \int \hat{S} \psi^* \hat{f} \hat{S}\psi dq = \int \psi^* \hat{S^*}^T \hat{f} \hat{S} \psi dq = \int \psi^* \underbrace{\hat{S}^+\hat{f}\hat{S}}_{\text{фактически новый оператор в представлении Гейзенберга}} \psi dq $$
$$ \hat{f}(t) = \hat{S}^{-1}f_ш \hat{S}$$
\par Оператор в новом представлении - это функция времени, можем записать уравнение движения.
$$\frac{\partial \hat{f}}{\partial t} =\frac{i}{\hbar}\hat{H}\hat{S}^+\hat{f_ш}\hat{S} -\frac{i}{\hbar}\hat{S}^+\hat{f_ш}\hat{S}\hat{H} = \frac{i}{\hbar}[\hat{H}\hat{f}] $$
\par Если есть базис функций, который диагонализирует $\hat{H}$, то есть фактически собственные функции этого оператора $\hbar{H}\psi_n =E_n \psi_n  $, то фактически временная зависимость снимается с уравнения на эти функции $\psi_n = e^{-\frac{i}{\hbar}E_n t}\varphi_n (q)$.
\par Если дана задача на УШ без начальных условий, тогда волновую функцию ищем как суперпозицию: $\psi = \sum_n a_n e^{-\frac{i}{\hbar}E_n t}\varphi_n (q)$. В норме в этом спектре все энергии ограничены снизу, а $\min_{n} E_n$ отвечает основному состоянию.
\par Дискретный спектр соответствует финитному движению (существует нормировочный интеграл), если никакая часть системы не уходит на $\infty$, то состояние называют \textit{связанным}. Волновые функции непрерывного спектра всегда уходят на бесконечность.
$$\psi = \int a_E e^{-\frac{i}{\hbar}Et} \psi_E(q)dE $$
$$|\psi|^2=\iint q_E q^*_E e^{\frac{i}{\hbar}(E^\prime-E)t} \psi_E(q) \psi^*_E(q) dE dE^\prime $$
\par При усреднении по времени и $t \rightarrow \infty$ получим $|\psi|^2 \rightarrow 0 $.