\newpage
\chapter{Четность состояния. Правило отбора по четности}
\par Нам потребуется оператор инверсии $\hat{P}\psi(\vec{r}) =\psi(-\vec{r})$. Его собственные числа $\pm 1$, т.к. если подействовать данным оператором два раза, то получится $P^2 = 1$. Следующее свойство: $[\hat{P} \hat{\vec{L}}] =0$.
\par Рассмотрим $I = \int \psi^*_n \hat{f}\psi_m dq $ и применим оператор инверсии, т.е. $\vec{r} \rightarrow{} -\vec{r} $, тогда 
\begin{equation*}
I = 
 \begin{cases}
    -I $ \rightarrow $ I=0
    \\
    +I
 \end{cases}
\end{equation*}
\par Получили так называемое \textbf{правило отбора}. //куча пропущенных слов// Отбор возможных переходов собственных функций. Обозначения: $g$ - четное состояние (от нем. \textit{gerade}) и $u$ - нечетное (от нем. \textit{ungerade}).
//здесь точно было пару примерчиков на лекции//
\par Действие оператора иверсии в декартовых и сферических координатах соответственно:
\begin{equation*}
\left.\begin{cases}
 x \rightarrow -x
 \\
 y \rightarrow -y
 \\
 z \rightarrow -z
\end{cases} 
\quad \begin{equation*}
    \left.\begin{cases}
    r \rightarrow  -r
     \\
   \varphi \rightarrow  \varphi + \pi
     \\
    \theta \rightarrow  \pi - \theta
    \end{cases} 
    \end{equation*}
\end{equation*}
\par Применим его к функциям:
$$P^m_l(cos\theta)e^{im\varphi} \longrightarrow (-1)^m e^{im\varphi}(-1)^{l-m}P^m_l(cos\theta)$$
\par Получается, что $P=(-1)^l$.