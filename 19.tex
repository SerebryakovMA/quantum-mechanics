\newpage
\chapter{Вариационный метод в квантовой механике}
\par Написали УШ, можем написать и функционал для него. Рассмотрим первую его вариацию
$$\delta \int \psi^* (\hat{H} - E)\psi dq = 0$$
\par $\psi$-функция комплексная, значит, есть две независимые действительные функции, отдельно будем варьировать по $\psi$ и по $\psi^*$. Выделим линейную часть
$$ \int \delta \psi^* (\hat{H} - E)\psi dq = 0 $$
\par Т.к. выполняется для любых отклонений $\delta \psi^*$, то получим УШ $\hat{H}\psi= E\psi$. Или же $\delta \int \psi^* \hat{H} \psi dq = 0$ при условии $\int |\psi|^2 dq = 1$ получим коэффициент - множитель Лагранжа.
\par $min \int \psi^* \hat{H} \psi dq$ - $1^{ое}$ (наименьшее/наинизшее) собственное значение энергии. Соответствующая $\psi_0$ - волновая функция основного состояния, остальные $\psi_{n>0}$ дадут экстремумы функционала, но не минимум. Требуются условия ортогональности каждых из следующих функций.
\par Подытожим: если $\psi_0, \psi_1...\psi_{n-1}$ - первые n состояний по возрастанию жнергий, то $\psi_n$ даст минимум $\int \psi^* \hat{H} \psi dq$ при условиях $\int |\psi|^2 dq = 1$ и $\int \psi^* \psi_m dq$ ($m=0,1...n-1$)
\begin{theorem} Волновая функция $\psi_0$ основного состояния не обращается в нуль (не имеет узлов) ни при каких конечных координатах.
\par
\proof Пусть не так, тогда возникает противоречие с принципом минимума из вариационного принципа, подробное доказательство под звездочкой.
\textbf{Теорема доказана}
\end{theorem}
\begin{corollary} Нижний уровень не вырожден
\par
\proof Возьмем две собственные функции, соответствующие нижнему уровню $E_0$, тогда их линейная комбинация так же будет являться собственной функцией $E_0$. Выбирая связь констант в ЛК всегда можем получить ноль и противоречие с предыдущей теоремой об отстутствии узлов.
\end{corollary}
\textbf{Следствие доказано}
\par Небольшое замечание:
$$\int \psi^* \hat{H} \psi dq = \int (-\frac{\hbar^2}{2m} \psi^* \nabla^2 \psi+U|\psi|^2)dq = ...$$
\par Интегрируем по частям и учитываем следующее тождество $div(\psi^* \nabla \psi)= |\nabla\psi|^2 + \psi^* \nabla^2 \psi$, результат:
$$...= \int \left(\frac{\hbar^2}{2m} |\nabla\psi|^2 + U |\psi|^2 \right)$$
\par Предположение достаточно быстрого спадания на бесконечности, т.е. спектр дискретный и $\int div(\psi^* \nabla \psi) = 0$. Если потенциал не обращается в бесконечность, можем выбрать нижнюю грань и спектр волновой функции будет положительно определен.