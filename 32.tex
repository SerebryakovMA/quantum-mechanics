\newpage
\chapter{Оператор спина. Уравнение Паули (нет в билетах этого года)}
\par По курсу еще не дошли до данной темы.
\par Спиновое квантовое число $m_s$ описывает направление вращения электрона в магнитном поле - по часовой стрелке или против. На каждой орбитали может находиться только два электрона: один со спином +½ другой -½.
Квантовые числа для первых трех энергетических уровней:

На первом уровне (n=1) есть только s-орбиталь, на которой может находиться только 2 электрона со спинами +1/2 и -1/2. Это справедливо для s-орбитали любого уровня: 1s; 2s; 3s…

На втором энергетическом уровне (n=2) есть уже две орбитали s; p. 
На третьем (n=3) - три орбитали: s, p, d. и т.д. С каждым новым энергетическим уровнем добавляется новая орбиталь.

Для 2p-орбитали существует три пространственных ориентации (формы облака), на каждой из которых может находиться по два электрона. Т.е. на втором энергетическом может находиться не более 6 p-электронов.

Для 3d - максимум 10 d-электронов и пять форм облаков.

Главные энергетические уровни отличаются энергией. Чем выше уровень - тем выше энергия. С другой стороны, различные орбитали одного и того же уровня также обладают разной энергией:
Энергия электронов на орбитали 2p выше, чем на 2s
Энергия электронов на орбитали 3p выше, чем на 3s
Энергия электронов на орбитали 3d выше, чем на 3s
Энергия электронов на орбитали 3d выше, чем на 3p
Что же касается электронов "внутри орбиталей", то их энергии одинаковы.

\par Принцип Паули: электроны располагаются так, что каждый из них имеет строго определённый набор квантовых чисел, в атоме не может быть даже двух электроновсо всеми четырьмя одинаковыми квантовыми числами.

Правило Хунда определяет порядок заполнения орбиталей определённого подслоя и формулируется следующим образом: модуль суммарного значения спинового-квантового числа электронов данного подслоя должен быть максимальным.

Это означает, что в каждой из орбиталей подслоя заполняется сначала один электрон, а только после исчерпания незаполненных орбиталей на эту орбиталь добавляется второй электрон.