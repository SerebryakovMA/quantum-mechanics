\newpage
\chapter{Операторы физических величин и матрицы}
\par Примеры операторов физических величин рассмотрены в предыдущем вопросе.
\section{Матричная формулировка квантовой механики}
\par Если мы фиксируем какое-то представление, а это представление дискретно в том смысле, что мы раскладываем по собственным функциям дискретного спектра, то каждому оператору в этом представлении можем поставить в соотсествие некоторую матрицу.
\par Пусть есть разложение произвольной $\psi$-функции по некоторому базису (с коэффициентами) - полному дискретному набору собственных функций какого-то оператора - $\psi = \sum_{n} a_n\psi_n$, для перехода к новому представлению необходимо верно определить вычисление средних. Несложно видеть, что общее определение может быть записано в следующем виде:
$$\overline{f}=\sum_{nm}a^*_n a_m f_{nm}(t) $$
\par  Причем матричный элемент $f_{nm}(t) = \int \psi^*_n \hat{f}\psi_m dq = f_{nm}e^{iw_{nm}t}$, здесь частота перехода $w_{nm}=\frac{E_n -E_m}{\hbar}$, а $f_{nm}=\int a^*_n \hat{f}\psi_m dq$. Предполагаем, что функции $\psi_n$ и $\psi_m$ соответствуют определенным энергиям, тогда на каждой "сидит" временная экспонента $ \sim e^{\frac{i E_j}{\hbar}} $, которую спокойно вынесли из-под интеграла.

\begin{remark}
\par \textit{Почему можем сказать, что формула записана в представлении Гейзенберга? Вернемся к общему определению среднего. Здесь $a^*_n$ и $ a_m$ играют роль волновой функции (мы так это записали, ввели без зависимости от времени), а вся зависимость от времени "сидит" \, в матрице $f_{nm}$, которая теперь играет роль оператора. В этом смысле запись пободна представлению Гейзенберга, но мы могли бы сделать и по-другому: временную зависимость "отправить"\, в $a^*_n$ и $ a_m$, получили бы некоторое другое выражение, но в том же духе. То есть этот экспоненциальный фактор можно либо оставить на матрице f, либо отправить на коэффициенты, это сейчас не важно.}
\end{remark}
\par Получившуюся \textit{f} естественно называть матрицей физической величины \textit{f}. Как определить производную по времени? Как и раньше, через производную по времени от среднего \textit{f}:
$$\overline{{\dot{f}}} = \dot{\overline{f}} = \sum_{mn} a^*_n a_m \dot{f}_{nm}(t) $$
\par где $\dot{f}_{nm}(t) = i w_{nm} f_{nm}(t)$. 
\begin{remark}
\par \textit{Что было бы с $f_{nm}$, если бы мы работали в базисе собственных функций оператора $\hat{f}$? Естественно, была бы диагональной, но мы этого не предполагаем.}
\end{remark}
\par Можем определить матрицу жрмитово сопряженного оператора, используя определение транспонированного оператора: $(f^*)_{nm} = \int \psi^*_n \hat{f^*}^T \psi_m dq = \int \psi_m f^* \psi^*_n dq$. Тогда
$$(f^*)_{nm}=(f_{mn})^*$$
\par Для вещественной физической величины $f_{nm}=f^*_{mn}$.
\section{Умножение матричных операторов}
\par Как с точки зрения матриц устроено умножение операторов? Распишем действие оператора на волновую функцию:
$$\hat{f} \psi_n = \sum_{m} f_{mn}\psi_{m}$$
\par А теперь рассмотрим действие произведения операторов, используя вышеприведенную формула сначала для оператора $\hat{g}$ и "протаскивая" \, оператор $\hat{f}$ (ведь он действует на волновую функцию), а потом и для $\hat{f}$:
$$\hat{f} \hat{g}\psi_n = \hat{f} \sum_k g_{kn}\hat{f}\psi_k = \sum_{km}g_{kn}f_{mk}\psi_m = \sum_m (fg)_{mn} \psi_m$$
\par Причем $f_{mk}g_{kn} = (fg)_{mn}$ - обычное матричное произведение. Пусть у нас есть задача на поиск собственных значений некоторого оператора $\hat{f}$: $\hat{f}\psi = f\psi$. Подставим разложение собственной функции $\psi = \sum_m C_m \psi_m$, тогда получим:
$$\sum_m C_m \hat{f} \psi_m = f \sum_{m}C_m \psi_m$$
\par Скалярно домножим на $\psi^*$ и проинтегрируем по пространству. Результат: 
$$ \sum_m f_{nm} C_m = f C_n$$
\par Как видно, задача на поиск собственных значений в матричном представлении сводится к вычислению собственных значений и векторов матрицы \textit{f}. Согласно линейной алгебре, для существования нетривиального решения (т.к. есть линейно зависимые строки)
 $det(f_{nm}-f_m\delta_{nm})=0 \rightarrow $ получим некоторое количество собственных значений. В базисе собственных функций матрица диагональна и представима в виде:  $f_{nm}=f_m\delta_{nm} $. Часто используют \textbf{обозначения Дирака}:
$$ f_{nm} \equiv < n|f|m> $$
$$ <n|m> = \int \psi^*_n \psi_m dq $$
\section{След и его свойства (полезное дополнение)}
\par \textbf{След матрицы} - это сумма всех ее диагональных элементов. Обозначается как \textbf{Sp} (от нем. \textit{Spur} — след) или \textbf{tr} (от англ. \textit{trace} — след).
\par 1\textdegree. $Spf=\sum_n f_{nn}$
\par 2\textdegree. $Sp(\hat{g}\hat{f})=Sp(\hat{f}\hat{g})$
\par 3\textdegree. $Sp(abc)=Sp(cab)$
