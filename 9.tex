\newpage
\chapter{Непрерывный и дискретный спектр. Вырождение уровней}
\par По определению о \textit{вырождении уровней} говорят, если одному собственному значению оператора соответствует несколько функций. Любая их линейная комбинация будет соответствовать этому значению, выделим такой набор $f_n: \psi_1{,}\psi_2{...}\psi_m{,}$. Предположим, что они не ортогональны, пытаемся выбрать новую функцию $\psi_2^\prime = \sum_i c_i \psi_i$, т.ч. она будет ортогональна $\psi_1$ ($(\psi_2^\prime{,}\psi_1) =0$) - получаем условие на $c_i$. Далее можем выбрать $\psi_3^\prime$, которая будет ортогональна и $\psi_1$, и $\psi_2^\prime$, продолжаем процесс ортогонализации. //послушать, что еще говорил//
\par \begin{theorem} Пусть f и g - одновременно измеримые физические величины, тогда их операторы коммутируют: $[\hat{f}{,}\hat{g}]=0$.
\par \proof  Если величины одновременно измеримы, то у них есть общая системы собственных функций, причем если она существует, классический прибор позволяет определить в $\psi_n$ некие значения величин $f_n{,}g_n$, если не существует, соответствующие значения $f_n{,}g_n$ измерить одновременно невозможно. Итак, 
$$ \hat{f}\hat{g}\psi_n = \hat{f} g_n \psi_n = f_n g_n \psi_n $$
$$ \hat{g}\hat{f}\psi_n = \hat{g} f_n \psi_n = f_n g_n \psi_n $$
\par Т.к. соотношения выполняются для $\forall \psi_n$, то $[\hat{f}{,}\hat{g}]=0$. \textbf{Теорема доказана.}
\end{theorem}
\par \begin{theorem} Вырождение уровней возникает, если есть две сохраняющиеся физические величины, операторы которых некоммутативны.
\par \proof  Дана $\psi$ - волновая функция некоторого стационарного состояния, в котором вместе с энергией имеет определенное значение величина f. Тогда $\hat{g}\psi$ не совпадает с самой волновой функцией, но дает собственные функции оператора $\hat{H}$, причем сама $\psi$ - тоже СФ $\hat{H}$.Величина сохраняется, поэтому она коммутирует с гамильтонианом:
$$ \hat{H} (\hat{g}\psi) = \hat{g}\hat{H}\psi = E \hat{g}\psi$$
\par E здесь - собственное число, т.е. получается, что E соответствует как волновой функции, так и $\hat{g}\psi$, а значит, возникает вырождение уровней. \textbf{Теорема доказана.}
\end{theorem}
\par \begin{theorem} (Обратная для Теоремы 1) Если операторы физических величин коммутируют, то у них есть общая система собственных функций, т.е. величины одновременно измеримы.
\par \proof $[\hat{f}{,}\hat{g}]=0$, отсюда $\hat{f}\hat{g}=\hat{g}\hat{f}$. Запишем в виде матричного произведения: 
$$\sum_{k} f_{mk}g_{kn} = \sum_{k} g_{mk}f_{kn} $$
\par Пусть набор собственных функций, в котором мы записываем матрицы - есть собственные функции оператора $\hat{f}$, тогда f - диагональная матрица: $f_{mk}=0$ при $m \ne k$, а значит, 
$$ g_{mn}(f_m - f_n)=0$$
\par Если все $f_i$ разные, то $\forall m \ne n$ выполняется $f_m-f_n\ne0 \rightarrow g_{mn} =0$, т.е. матрица может быть выбрана диагональной в данном базисе, тогда $\psi_n$ - с.ф. и для $\hat{g}$. Если есть вырождение, всегда можно сделать поворот базиса вырожденных решений т.ч.$g_{mn} =0$. \textbf{Теорема доказана.}
\end{theorem}
\par \textnormal