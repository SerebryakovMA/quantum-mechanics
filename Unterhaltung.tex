\newpage

\chapter{То что знали, но всегда хотели узнать}
\par Разные интересные факты и их доказательства, а также расписанный ход мысли. Данная глава выступает всеобщим протестом против слов: очевидно, не трудно показать и тд...
\section{Какой знак правильный?}
\par Старый пряник однажды спросил у хорошего, но малознающего студента: \textit{Почему у Вас уравнение Шрёдингера написано так: $ \mathfrak{ -i h \frac{\partial \psi(t)}{\partial t} = \hat{H}\psi } $}. Не очевидно, что принято записывать уравнение Шрёдингера так: $ \mathfrak{ i h \frac{\partial \psi}{\partial t} = \hat{H}\psi } $.
\par На самом деле обе записи верны: сменой направления течения времени они сводятся друг к другу, то есть при замене $\mathfrak{t \rightarrow -t}$. Напомню, проблема с течением времени решается лишь только в некоторых областях физики при помощи добавления вероятности происхождения того или иного события, а также при помощи причнно-следственной связи.
\section{Преподаватель на зачёте был не нужен}
\par Все совпадения с реальными зачётами, где студенты понимали вопрос преподавателя, искажая его смысл, остаются глубоко в душе и требуют ответа. Тема этой секции - уравнение непрерывности: $$\mathfrak{\frac{\partial |\psi|^2 }{\partial t} + div \! \left( \ \! \vec{j} \ \! \right) = 0 }$$
\begin{center}
    \textit{ Как вывести в общем виде, что $\mathfrak{\frac{\partial |\psi|^2 }{\partial t}}$ это есть дивергенция какого-то вектора и только? } 
\end{center}
\par Прежде чем решать такую сложную и общую задачу надо пытаться решать для более простой с гамильтанианом частного вида: $\mathfrak{ \hat{H} = \frac{\hat{p}^2}{2m}}$. При решении можно получить следующие важные сведения: во-первых, от импульса $\mathfrak{\hat{\vec{p}} = -i h \vec{\nabla}} $ нам важно  $\mathfrak{\hat{\vec{p}} \sim \vec{\nabla}}$, поэтому для вывода уравнения непрерывности нам нужно следить за степенями $\mathfrak{\hat{\vec{p}}}$. Во-вторых, Гамильтониан применённый к пси функции должен выдавать скаляр умноженный на пси функцию, никак не вектор, следствием этого будет требование некторых констант быть векторами. Из сказанного выше выпишем гамильтаниан общего вида, разложив его в полином от, учитывая что комутатор импульса и константа перед ними не коммутируют: $\mathfrak{\hat{\vec{p}}}$:
$$ \mathfrak{ \hat{H} = \sum_{s=0}^{k} A_s  \hat{\vec{p}}^{ \ s} + \sum_{s=0}^{k}  \hat{\vec{p}}^{ \ s} B_s  }  $$
Для нечётных $\mathfrak{s}$ важное свойство $\mathfrak{A_s}$: $ \mathfrak{ A_{2n+1} \equiv{} \vec{A}_{2n+1}}$. Приведу вывод для коэффициентов, хотя, зная вид оператора импульса, можно сказать, что это уже было выведено. Воспользуемся знанием, что $\mathfrak{\hat{{H}}}$ - эрмитов оператор: $\mathfrak{ \left(  \hat{H} \psi, \psi \right) = \left( \psi, \hat{H} \psi \right)} $
 $$ \mathfrak{ \left(   \psi, \hat{H} \psi \right) = \int \psi^* \hat{H} \psi \ dq = \int \psi^* \left( \sum_{s=0}^{k} A_s   \hat{\vec{p}}^{ \ s} \right)  \psi \ dq  =A_0 \int |\psi|^2 \ dq + \int \psi^* \left( \sum_{s=1}^{k} A_s   \hat{\vec{p}}^{ \ s} \right)  \psi \ dq} = $$
 $$ \mathfrak{ = A_0 + \int \psi^* \left( \sum_{s=0}^{k-1} A_{s+1}   \hat{\vec{p}}^{ \ s+1} \right)  \psi \ dq = A_0 + \int \psi^* \hat{\vec{p}}  \left( \sum_{s=0}^{k-1} A_{s+1}   \hat{\vec{p}}^{ \ s} \right)  \psi \ dq }$$
\par Далее потребуется воспользовать знанием что импульс это $\vec{\nabla}$ тогда можно записать подытегральную функцию как $ \mathfrak{\nabla \left( \psi^* \left( \sum_{s=0}^{k-1} A_{s+1}   \hat{\vec{p}}^{ \ s} \right)  \psi \right) - \hat{\vec{p}} \psi^*   \left( \sum_{s=0}^{k-1} A_{s+1}   \hat{\vec{p}}^{ \ s} \right)  \psi }$. По теореме Остроградского-Гаусса получим:
$$ \mathfrak{ A_0 + \oint \psi^* \left( \sum_{s=0}^{k-1} A_{s+1}   \hat{\vec{p}}^{ \ s} \right)  \psi \ dS - \int \hat{\vec{p}} \psi^*   \left( \sum_{s=0}^{k-1} A_{s+1}   \hat{\vec{p}}^{ \ s} \right)  \psi \ dq  = A_0 - A_1 \int \hat{\vec{p}} \psi^* \psi \ dq - } $$
$$ \mathfrak{ - \int \hat{\vec{p}} \psi^*   \left( \sum_{s=0}^{k-2} A_{s+2}   \hat{\vec{p}}^{ \ s+1} \right)  \psi \ dq }$$
\par Далее действуем аналогично и получаем при некоторых условиях (чтобы интегралы по замкнутым поверхностям были нулями) получаем следующую формулу:
$$ \mathfrak{\int \psi^* \left( \sum_{s=0}^{k} A_s   \hat{\vec{p}}^{ \ s} \right)  \psi \ dq = \int \psi \left( \sum_{s=0}^{k} (-1)^s A_s    \hat{\vec{p}}^{ \ s} \right) \psi^*  \ dq} $$
\par из определения сопряженного оператора следует, что справа написан комплексно сопряжённый оператор к $\mathfrak{\hat{H}}$, поэтому $\mathfrak{ \sum_{s=0}^{k} A_s^*   \hat{\vec{p}}^{*  s} = \sum_{s=0}^{k} (-1)^s A_s \hat{\vec{p}}^{ \ s} }$. Делаем вывод - определённый нами оператор импульса подходит под данных критерий, т.к. все его чётные степени при комплексном сопряженнии не меняют знак, а нечётные - меняют. Определив так импульс, мы требуем от чисел (векторов) $\mathfrak{A_s}$ действительность. Аналагично проделаем тоже самое с другой частью, в  выводе главное следить за порядком действия $\mathfrak{ B}$ и $\mathfrak{ \hat{\vec{p}} }$.  Сейчас у нас всё готово для доказательства в силу уравнения Шрёдингера:
$$ \mathfrak{ \frac{\partial |\psi|^2 }{\partial t} = \psi \frac{\partial \psi^*}{\partial t} + \psi^* \frac{\partial \psi}{\partial t} = \frac{i}{h} \left(  \psi^* \hat{H} \psi - \psi \hat{H}^* \psi^* \right)} $$
\par Из эрмитовости $\mathfrak{\hat{H}}$ (для проверки припишите слева ещё одну волновую функцию и воспользуйтесь определением сопряжённого оператора) получим:
$$ \mathfrak{ \frac{i}{h} \left(  \psi^* \hat{H} \psi - \psi \hat{H}^* \psi^* \right) = }$$
\par Из-за множителя $\mathfrak{(-1)^s}$ Останутся лишь нечётные элементы ряда, которые удвоятся. Выделим один $\mathfrak{\hat{\vec{p}}}$ и распишем его через оператор набла, не забудем что нечётные $\mathfrak{A_s}$ - действительные вектора , результат: ($\mathfrak{[x]}$ - целая часть числа $\mathfrak{x}$)
$$ \mathfrak{ 2 \vec{\nabla} \left( \sum_{s=0}^{[\frac{k-1}{2}]} \vec{A_{2s+1}} \hat{\vec{p}}^{ \ 2s} |\psi|^2 \right)  \equiv -div \! \left( \ \! \vec{j} \ \! \right) } $$
$$ \mathfrak{ \vec{j} = -2 \sum_{s=0}^{[\frac{k-1}{2}]} \vec{A_{2s+1}} \hat{\vec{p}}^{ \ 2s} |\psi|^2  } $$
\par {\gothfamily Kvant Mech}



